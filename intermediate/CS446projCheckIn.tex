\documentclass[11pt,letterpaper]{article}
\oddsidemargin 0in
\evensidemargin 0in
\textwidth 6.5in
\topmargin -0.5in
\textheight 9.0in
\usepackage{hyperref}
\usepackage{mathptmx}
\usepackage{graphicx}
\usepackage{natbib} % for references
\usepackage[usenames,dvipsnames]{xcolor}
\newcommand{\blue}[1]{\textcolor{RoyalBlue}{#1}}
\newcommand{\fillme}[1]{\blue{\texttt{[Insert #1]}}}
\newcommand{\instructions}[1]{\blue{\textit{#1}}}
% uncomment the next two lines if you want the instructions to disappear.
%\renewcommand{\instructions}[1]{}
%\renewcommand{\fillme}[1]{}

\begin{document}

\title{CS446 Class Project: \fillme{Your Project}}
\author{\fillme{Your Names (Your NetIDs)}}
\maketitle



\instructions{If you are taking CS446 for 4 hours credit, you need to
  do a research project. This is a template for the intermediate
  report (the check in),
  but this should also give you a start on the final report.
The template for the final report is at
\url{http://courses.engr.illinois.edu/cs446/Projects/CS446projCheckIn.tex}
(or
\url{http://courses.engr.illinois.edu/cs446/Projects/CS446projCheckIn.pdf}
for the pdf). Uncomment the \texttt{$\backslash$renewcommand{$\backslash$instructions}[1]\{\}}
 and \texttt{$\backslash$renewcommand{$\backslash$fillme}[1]\{\}} lines in the preamble of the template if
 you want the instructions to disappear.
% \begin{verbatim}
% 
% \end{verbatim}
}


\begin{abstract}
\instructions{Very briefly, summarize your task, your model and your main
results} 
\end{abstract}


\section{Introduction} 
\label{sec:introduction}
\instructions{This should be a brief outline of the paper -- use plain English, no math. Note that you should be able to write most of this section before you actually perform any experiments. First, define and motivate your task: what are you trying to learn, and why is this an important task? Second, define what kind of a machine learning problem this requires you to solve (binary/multiclass classification, ranking, ....). What is an appropriate baseline model for this task? What kind of model are you proposing? 
Briefly summarize the assumptions your model makes. Finally, describe the hypotheses you wish to test. These are typically statements of the form ``we expect model/features A to perform better on this task than model/features B''.
Outline how your experiments will evaluate these hypotheses (comparisons of different models, ablation studies, learning curves, oracle experiments... ). }

\section{Background}
\label{sec:background}
\instructions{Summarize and discuss related work that you are building on: this requires you to find, read and cite a few research papers. This is also something you can get started on as soon as you have settled on a task.} 

\section{Task and Data}
\label{sec:taskAndData}
\instructions{Now describe the task and data in more detail.}

\subsection{The Task}
\label{sec:task}
\instructions{Now, try to formalize your task as a classification/ranking/... problem. Introduce mathematical/formal notation as necessary. How do you evaluate models, or measure success?}

\subsection{The Data}
\label{sec:data}
\instructions{Describe the data you use to train and evaluate your models. Describe where you got it from (include references/citations to published works, or URLs!). Describe and give examples for the features that you have access to.} 


\section{The Models}
\label{sec:models}

\subsection{Baseline Models}
\label{sec:baseline-models}
\instructions{In order to know how difficult the task is and how well we are doing, we need to know how well a suitable baseline model would perform. Define a baseline model for your task. This may not necessarily be a learned model.}

\subsection{Existing Models}
\label{sec:existing-models}
\instructions{If people have worked on this task before, summarize (and cite) some of the existing models} 

\subsection{Proposed Model(s)}
\label{sec:proposed-models}
\instructions{Your models and your procedure for learning them go here. Describe both in detail, even if the learning procedure is standard.}

\section{Experiments}
\label{sec:experiments}

\subsection{Experimental Hypotheses}
\label{sec:exper-hypoth}
\instructions{Summarize the hypotheses (research questions) your experiments are designed to test (address). (Note that some of these hypotheses may emerge as you keep working on a problem; you will not necessarily have come up with all the questions you wish to address before you have started building a models for the specific task.}

\subsection{Experimental setup}
\label{sec:experimental-setup}
\instructions{Define test/training/dev data splits, describe how you tuned performance. Describe and your evaluation metric, and define it mathematically.
List the models you will evaluate. Cite any existing tools or software you use to perform your experiments; describe what you implemented yourself. Describe how you obtained the features used by each of the models.}

\subsection{Experimental results}
\label{sec:experimental-results}
\instructions{Now give the actual experimental results (use figures/tables/graphs as appropriate), and discuss whether they verify or falsify your hypotheses. How important are the various features your models use (consider ablation studies). How robust are your results? (Look at learning curves, or the variance when you perform cross-validation). Can you perform an error analysis?}

\section{Conclusion}
\instructions{Summarize your findings, and discuss their implications, e.g. for future work, or for related tasks. Discuss also the shortcomings of your proposed approach. }. 

\section*{\instructions{Bibliography}}
\instructions{Don't forget to create your own .bib file. If you call it {\tt mybib.bib} and put it in the same directory as this {\tt .tex} file, add {\tt$\backslash$bibliography\{mybib\}} before {\tt$\backslash$end\{document\}}
}
\bibliographystyle{natbib}  
%\bibliography{Your .bib file}

\section*{Your current to-do list}
\instructions{This should be an updated version of your initial to-do
  list. Compare what you have done with what remains to be done. If you have a group
  project: who will do what? Set yourself deadlines. Here are a few
  items that might appear on your to-do list}
\paragraph{Done}
\begin{enumerate}
\item \fillme{...}
\item \fillme{...}
\end{enumerate}
\paragraph{Left to do}
\begin{enumerate}
\item \fillme{...do you have data?}
\item \fillme{...do you know related work? (have you got the
    references  for your .bib file?)}
\item \fillme{...what algorithm will you use? do you need to implement
    this yourself, or will you use an off-the-shelf package?} 
\item \fillme{...what experiments do you plan to run?}
\item \fillme{...and don't forget to allocate time for the writeup!} 
\end{enumerate}

\section*{\instructions{Bibliography}}
\instructions{If you need references for the background section, don't forget to create your own .bib file. If you call it {\tt mybib.bib} and put it in the same directory as this {\tt .tex} file, add {\tt$\backslash$bibliography\{mybib\}} before {\tt$\backslash$end\{document\}}
}
\bibliographystyle{natbib}  
\end{document}
